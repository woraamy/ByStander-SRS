\chapter{Software Architecture Design}
\label{chap:software-architecture-design}

% TIP: Describe how you design your application using Unified Modelling
% Language (UML). There should be at least two diagrams that describe the
% software architecture. You may add additional or remove unnecessary diagrams.
% However, there needs to be a coherency between them at the end.

\section{Domain Model}
\label{section:domain-model}

% TIP: Describe the business concept of your project. Showcase a
% domain model that captures the said concept.

\section{Design Class Diagram}
\label{section:design-class-diagram}

% TIP: Showcase a design class diagram for your project and explain
% how it works here. You can group classes into packages or layers to communicate your
% design better.

\section{Sequence Diagram}
\label{section:sequence-diagram}

% TIP: Sequence diagrams describe how the software runs at runtime.
% You do not have to create a sequence diagram for every scenario. However,
% there should be one for all the main ones.

% ChatGPT: Creating a sequence diagram for every use case is not
% strictly necessary, but it can be a valuable tool in certain situations. Sequence
% diagrams are particularly useful for illustrating the interactions between different
% components or objects in a system over time, showcasing the flow of messages
% or actions between them.

\section{Algorithm}
\label{section:algorithm}

% TIP: Optional, If you are working on a research project that proposes a new
% algorithm, you can describe your algorithm here. It can be in the form of
% pseudocode or any diagram that you deem appropriate.

\section{AI component}
\label{section:ai-component}

The AI component forms the core intelligence of the emergency response system, providing speech recognition, emergency guidance, text-to-speech capabilities, facility recommendations, and communication assistance. This document outlines each feature's implementation and deployment strategy.

\subsection{Feature 1: Emergency Voice Transcription Service}

    \subsubsection{Input/Output}
    \begin{itemize}
        \item Input:  Voice recording of user describing emergency in Thai
        \item Output: Transcribed text 
    \end{itemize}

    \subsubsection{Processing Logic}
    \begin{enumerate}
        \item Capture voice recording through device microphone
        \item Preprocess audio for noise reduction and normalization
        \item Send to Gowajee.ai API for Thai-optimized transcription
        \item Give the transcribed text back to process later on
    \end{enumerate}

\subsubsection{Model Used}
\begin{itemize}
    \item Primary Model: Google Speech-to-Text API with Thai language support
\end{itemize}

\subsection{Feature 2: Contextual Emergency Guidance Generation}

\subsubsection{Input/Output}
\begin{itemize}
    \item Input: Transcribed text 
    \item Output: Step-by-step emergency guidance text and related video references
\end{itemize}

\subsubsection{Processing Logic}
\begin{enumerate}
    \item Analyze extracted keywords for emergency classification
    \item If the situation user provided is not an emergency, return a message to the user
    \item Query emergency procedures database for relevant protocols
    \item Generate contextually appropriate guidance based on emergency type
    \item Structure output in clear, simple language optimized for crisis situations
    \item Tag relevant instructional video content where applicable
\end{enumerate}

\subsubsection{Model Used}
\begin{itemize}
    \item Primary Model: Fine-tuned Claude LLM (Haiku version)
    \item Training Dataset: Emergency procedures from Thai medical websites, first aid protocols, medical emergency keyword database
\end{itemize}

\subsection{Feature 3: Text-to-Speech for Guidance}

\subsubsection{Input/Output}
\begin{itemize}
    \item Input: Generated emergency guidance text
    \item Output: Clear audio instructions in Thai language
\end{itemize}

\subsubsection{Processing Logic}
\begin{enumerate}
    \item Optimize text for audio readability (punctuation, pauses)
    \item Process through TTS engine with emergency-appropriate voice profile
    \item Stream audio for immediate playback while buffering remainder
    \item Adjust speaking pace based on emergency type severity
\end{enumerate}

\subsubsection{Model Used}
\begin{itemize}
    \item Primary Model: Google TTS API with Thai language support
\end{itemize}

% \subsection{Feature 4: Emergency Facility Finder and Recommendations}

% \subsubsection{Input/Output}
% \begin{itemize}
%     \item Input: User's GPS location, emergency type, time of day
%     \item Output: Prioritized list of appropriate emergency facilities with contact information and routing
% \end{itemize}

% \subsubsection{Processing Logic}
% \begin{enumerate}
%     \item Query Google Maps API for nearby emergency facilities
%     \item Filter facilities based on emergency type compatibility
%     \item Apply multi-factor ranking algorithm considering:
%     \begin{itemize}
%         \item Geographic proximity
%         \item Facility specialization relevance
%         \item Current availability based on time
%         \item Historical response efficiency
%         \item Traffic conditions
%     \end{itemize}
%     \item Generate recommendation list with clear reasoning
%     \item Prepare facility contact information and routing guidance
% \end{enumerate}

% \subsubsection{Model Used}
% \begin{itemize}
%     \item Primary Model: Deepseek LLM with geospatial processing extensions
%     \item Integration: Google Maps API for location intelligence
%     \item Dataset: Comprehensive emergency facility database for Thailand with capabilities matrix
%     \item Algorithm: Weighted decision model with emergency-specific priority factors
% \end{itemize}

\subsection{Feature 5: Phone Operator Generation}

\subsubsection{Input/Output}
\begin{itemize}
    \item Input: Emergency type, user location, target facility, emergency details
    \item Output: Structured communication script for speaking with emergency operators
\end{itemize}

\subsubsection{Processing Logic}
\begin{enumerate}
    \item Identify easy-to-reference landmarks near user location via Google Nearby Search
    \item Select appropriate script template based on emergency type
    \item Structure information in operator-preferred sequence:
    \begin{itemize}
        \item Emergency type and severity
        \item Location with landmark references
        \item Victim status and key details
        \item Requested assistance type
    \end{itemize}
    \item Optimize language for clarity in high-stress situations
    \item Format script for easy reading during emergency call
\end{enumerate}

\subsubsection{Model Used}
\begin{itemize}
    \item Primary Model: Fine-tuned Deepseek LLM for script generation
    \item Integration: Google Maps API and Google Nearby Search
    \item Dataset: Script templates for different emergency types
\end{itemize}