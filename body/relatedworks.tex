\chapter{Literature Review and Related Work}
\label{chap:relatedworks}

In this chapter, describe other solutions/research that address the
same topic as your project. If you are working on a software project, create a
list of alternative solutions and analyze them in the competitor analysis section.
If you are working on a research project, describe your related work research in
the literature review section.

\section{Competitor Analysis}
\label{section:competitor-analysis}

% \begin{figure}[h]
%     \centering
%     \includegraphics[width=0.5\textwidth]{examples/asana-competitive-landscape.jpg}
%     \caption{Competitive Landscape by Asana}
% \end{figure}

These are the application with similar functionalities to ByStander

\begin{enumerate}
    \item \textbf{ICE - In Case of Emergency} \\
    The ICE application is designed primarily to store emergency contacts and medical information, such as allergies and emergency contacts, for quick access during an emergency. While this is helpful for ensuring that medical details are available in emergencies, it lacks the real-time guidance that ByStander offers. ByStander, with its AI-driven system, not only helps identify the appropriate emergency contact but also provides dynamic, step-by-step guidance during an emergency situation, enhancing decision-making under stress.    
    \item \textbf{First Aid by American Red Cross} \\
    The Red Cross app offers clear instructions for first aid procedures, ranging from CPR to treating burns or fractures. While the instructions are easy to follow and offer a valuable resource for immediate medical emergencies, the app does not address other critical aspects of emergency response, such as choosing the right emergency service or location-based routing. ByStander distinguishes itself by offering real-time, optimized routing and expert-powered guidance, beyond just first aid.
    \item \textbf{911 Assistance Apps} \\
    Several local and national emergency assistance apps allow users to call emergency services and provide real-time location data to responders. These apps, while useful in urgent situations, generally lack the intelligent, context-sensitive decision-making that ByStander offers. For instance, ByStander tailors its response to the nature of the emergency and the user’s location, offering step-by-step guidance and helping to reduce cognitive strain under stress.
\end{enumerate}

\begin{table}[htp]
\tiny
\centering
\begin{tabular}{|p{2.2cm}|p{2.2cm}|p{2.2cm}|p{2.2cm}|p{2.2cm}|}
\hline
\textbf{Feature}                           & \textbf{ICE - In Case of Emergency}                     & \textbf{First Aid by American Red Cross}            & \textbf{911 Assistance Apps}                        & \textbf{ByStander}                                     \\ \hline
\textbf{Emergency Contact Storage}          & Yes                                               & No                                            & Yes                                            & No                                               \\ \hline
\textbf{Real-Time Location-Based Routing}   & No                                                & No                                            & Yes                                            & Yes                                              \\ \hline
\textbf{Real-Time Emergency Guidance}       & No                                                & Yes (first aid only)                          & No                                             & Yes                                              \\ \hline
\textbf{Step-by-Step Instructions}          & No                                                & Yes                                           & No                                             & Yes                                              \\ \hline
\textbf{AI-Driven Guidance}                 & No                                                & No                                            & No                                             & Yes                                              \\ \hline
\textbf{Expert-Powered Content}             & No                                                & No                                            & No                                             & Yes                                              \\ \hline
\textbf{Emergency Service Contact}          & No                                                & No                                            & Yes                                            & Yes (optimized, based on location and type)      \\ \hline
\textbf{Scope of Emergency Types}           & General (contact info)                            & Medical emergencies (first aid focused)       & General emergencies (mainly police/fire)       & Wide range (medical, accident, fire, etc.)       \\ \hline
\textbf{User-Friendliness}                  & High                                              & High                                          & High                                           & High                                             \\ \hline
\textbf{Target User}                        & General public, with emphasis on contact info     & Individuals needing first aid instructions   & Individuals needing to contact emergency services & General public, with tailored guidance for all types of emergencies \\ \hline
\textbf{Customization}      & No                                                & No                                            & No                                             & Yes (context-sensitive, AI-based, expert content) \\ \hline
\end{tabular}
\caption{Comparison of Emergency Assistance Applications}
\label{tab:competitor-analysis}
\end{table}
    
\section{Literature Review}
\label{section:literature-review}


In emergency situations, panic often plays a critical role in exacerbating an individual's ability to respond effectively, potentially leading to poor decision-making and delays in seeking help. According to Aguirre (2005), panic can lead to a breakdown in communication, confusion, and hesitation, which worsens the outcomes in emergency scenarios. This psychological phenomenon is especially dangerous in high-stress situations, where quick and decisive action is crucial to minimize harm. Panic can disrupt normal cognitive processing, leading individuals to either freeze or make erratic decisions that are not based on a rational assessment of the situation \cite{aguirre2005emergency}. Research also indicates that emergency evacuations and crises are significantly impacted by social psychology and the widespread tendency to panic, which can delay the response efforts and hinder the coordination of emergency services \cite{aguirre2005emergency}.

In emergency medical contexts, panic-related cognitive impairment becomes even more pronounced. Foldes-Busque et al. (2017) highlight the prevalence of emergency department (ED) visits triggered by panic attacks, particularly in patients presenting with non-cardiac chest pain. Their study found that a substantial portion of ED visits can be linked to panic attacks, suggesting that individuals experiencing psychological distress may seek emergency care unnecessarily, further overwhelming medical resources. This highlights the need for interventions that can reduce panic and support individuals in managing emergency situations more effectively \cite{foldesbusque2017closer}.

Artificial intelligence (AI) has emerged as a promising technology in mitigating cognitive impairment during emergencies. AI offers the potential to enhance decision-making by providing real-time, actionable insights based on vast amounts of data, which can help individuals navigate stressful scenarios more effectively. Kirubarajan et al. (2020) reviewed the role of AI in emergency medicine, noting that AI technologies, such as machine learning and deep learning, can significantly improve the accuracy of diagnoses, particularly in the context of acute radiographic imaging and patient outcome predictions. These technologies help reduce cognitive overload by offering data-driven support that minimizes human error. Additionally, AI's ability to analyze data in real time could help in predicting emergency events or identifying critical issues, allowing for faster and more efficient interventions \cite{kirubarajan2020artificial}.

In the context of emergency response, mobile applications have become a key tool for enhancing real-time communication and decision-making. Several studies have demonstrated the benefits of mobile apps that utilize geolocation and other technologies to provide location-based emergency response. For instance, de Guzman and Ado (2014) developed a mobile emergency response application using geolocation for command centers, which allows emergency personnel to quickly identify the most critical situations based on geographic information. Similarly, Romano et al. (2016) explored how mobile apps could enable citizens to act as "human sensors" during emergencies by gathering real-time data from users on the ground. These applications help by offering step-by-step guidance for first responders and citizens, reducing the reliance on bystanders’ decision-making under stress and enhancing coordination among emergency teams \cite{deguzman2014mobile, romano2016designing}.

The integration of AI into mobile applications further amplifies their potential. Mobile apps that combine AI and machine learning capabilities can guide users through emergencies by providing personalized instructions based on real-time data, such as the user’s location, the nature of the emergency, and previous emergency patterns. As Kiran Grant et al. (2020) explain, AI has transformative potential in emergency medicine by enabling faster and more accurate decisions, such as predicting patient outcomes and assisting in diagnostic processes. AI-enabled mobile applications could significantly improve the way users respond to medical emergencies by offering tailored, real-time guidance, thus alleviating the cognitive load and reducing the impact of panic on decision-making \cite{kiran2020artificial}.

As research continues to evolve in both AI and mobile technology for emergency response, it is evident that these innovations have the potential to revolutionize the way individuals and emergency services handle crises. By reducing the cognitive impairments caused by panic and offering real-time, data-driven support, AI and mobile apps can significantly enhance emergency outcomes and improve response times \cite{kirubarajan2020artificial, kiran2020artificial}. The development of mobile applications that integrate AI, like the one proposed in this research, offers a comprehensive solution to the challenges posed by panic and cognitive overload in emergency situations.
