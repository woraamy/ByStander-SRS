\chapter{Literature Review and Related Work}
\label{chap:relatedworks}

In this chapter, describe other solutions/research that address the
same topic as your project. If you are working on a software project, create a
list of alternative solutions and analyze them in the competitor analysis section.
If you are working on a research project, describe your related work research in
the literature review section.

\section{Competitor Analysis}
\label{section:competitor-analysis}

% \begin{figure}[h]
%     \centering
%     \includegraphics[width=0.5\textwidth]{examples/asana-competitive-landscape.jpg}
%     \caption{Competitive Landscape by Asana}
% \end{figure}

Current emergency response applications, while valuable, present significant limitations in addressing the complex challenges faced 
by individuals during crisis situations. A comprehensive analysis of existing solutions reveals critical gaps that ByStander aims to 
overcome with its innovative approach.

\begin{enumerate}
    \item \textbf{JS100 Application (Android, iOS)} \\
    JS100 Application offers an SOS function that allows users to share their location during emergencies with a single tap. While the app effectively pinpoints a user's location for emergency services, it provides no guidance or contextual assistance during the emergency itself. This critical gap means users must rely solely on their own knowledge during stressful situations, potentially leading to poor decision-making when quick, informed actions are most needed. In contrast, ByStander not only facilitates emergency contact but delivers AI-driven, personalized guidance based on the emergency type, significantly improving outcomes by reducing panic and providing clear direction during critical moments.
    
    \item \textbf{First Aid by American Red Cross} \\
    The First Aid by American Red Cross app provides users with easy-to-follow instructions for a variety of first aid procedures, such as performing CPR, treating burns, fractures, and other common medical emergencies. While these instructions are clear and offer valuable assistance for immediate medical situations, the app falls short in addressing other critical aspects of emergency response. It does not provide guidance on selecting the correct emergency service or suggest the nearest facilities for assistance. In contrast, ByStander distinguishes itself by not only offering detailed first aid instructions but also providing real-time location-based routing and expert-powered guidance. This integration of dynamic response options enhances the app’s utility in a broader range of emergencies, beyond just first aid.

    \item \textbf{Emergency+ (Australia)} \\
    The Emergency+ app is a GPS-based service designed to help users locate the nearest emergency services in Australia. By using the user’s current location, it allows them to quickly reach the correct emergency service provider. The app displays essential details like the address and contact number for services like police, fire, and ambulance. However, it does not provide the in-depth, real-time guidance or the AI-driven emergency assistance that ByStander offers. ByStander takes it a step further by offering location-based guidance and optimized decision-making based on the nature of the emergency, providing both immediate and actionable support.

\end{enumerate}

ByStander addresses these limitations by integrating advanced technologies identified in current research with practical emergency response needs. Unlike existing applications, ByStander combines AI-driven decision support with location-based routing and expert-verified guidance to create a comprehensive emergency response system. Building on the findings of Kirubarajan et al. (2020) regarding AI's potential in emergency medicine, ByStander leverages artificial intelligence to provide real-time, actionable insights that can significantly reduce cognitive load during emergencies. This approach directly addresses the issues of panic-induced cognitive impairment highlighted in emergency response literature, offering users clear, contextually relevant guidance when they need it most.

\begin{longtable}{|p{3cm}|p{3cm}|p{3cm}|p{3cm}|p{3cm}|}
    \hline
    \textbf{Feature} & \textbf{ByStander} & \textbf{First Aid by American Red Cross} & \textbf{Emergency+ (Australia)} & \textbf{JS100 Application} \\
    \hline
    \textbf{First Aid Instructions} & Comprehensive guidance & Detailed but static instructions & Basic information only & None \\
    \hline
    \textbf{Emergency Service Selection} & AI-assisted selection & Not available & Manual selection & One-button Alert System \\
    \hline
    \textbf{Location-Based Routing} & Real-time routing to nearest appropriate facility & Not available & Basic GPS location sharing & GPS location sharing only \\
    \hline
    \textbf{AI-Driven Assistance} & Personalized guidance based on emergency context & Not available & Not available & Not available \\
    \hline
    \textbf{Panic Detection} & Automatic detection of user distress & Not available & Not available & SOS button only \\
    \hline
    \textbf{Step-by-Step Crisis Guidance} & Dynamic guidance adapting to situation changes & Static instructions only & Not available & Not available \\
    \hline
    \textbf{Community Expert Input} & Localized guidance from verified experts & General information only & Not available & Not available \\
    \hline
    \textbf{Stress-Resistant Interface} & Specifically designed for high-stress usability & Standard interface & Standard interface & Basic interface \\
    \hline
    \hline
    \textbf{Multiple Emergency Types} & Medical, accidental, and crime emergencies & Medical emergencies only & All types but limited guidance & NO specific categorization \\
    \hline
    \textbf{Area of available} & Thailand & United States & Australia & Thaland \\
    \hline
\caption{Comparison of Emergency Assistance Applications}
\label{tab:competitor-analysis}
\end{longtable}
    
\section{Literature Review}
\label{section:literature-review}


Emergency situations create unique cognitive challenges that can significantly impair an individual's ability to respond effectively. \cite{aguirre2005emergency} has documented how panic leads to communication breakdowns, confusion, 
and decision paralysis in emergency scenarios, highlighting the critical need for interventions that can offset these psychological limitations. 
This research underscores the importance of developing tools that can maintain rational decision-making capabilities even when users are experiencing extreme stress—a core design principle 
behind ByStander's interface and AI assistance system. \\

The prevalence of panic-induced decision-making in emergency contexts is further supported 
by \cite{foldesbusque2017closer}, who found that a substantial portion of emergency department 
visits for non-cardiac chest pain are triggered by panic attacks. This research highlights how psychological 
distress can lead to resource misallocation in emergency medical services, reinforcing the need for applications like ByStander that can help users make more informed decisions about when and how to seek emergency assistance. \\ 

Artificial intelligence offers promising solutions to these challenges. \cite{kirubarajan2020artificial} examined AI's role 
in emergency medicine, noting its potential to enhance diagnostic accuracy and reduce cognitive overload through data-driven support. 
Similarly, \cite{kiran2020artificial} highlighted AI's transformative potential in emergency medicine through faster and more accurate 
decision-making capabilities. These findings inform ByStander's AI-driven approach, which aims to provide users with personalized, 
real-time guidance based on emergency type, location, and historical patterns. \\ 

Voice recognition technology represents another critical component of emergency response systems. \cite{voice_recognition} 
explored speech-to-text recognition systems that can accurately convert spoken language into text, even in noisy environments. 
Their research demonstrates how such technology can be particularly valuable in emergency situations where manual input may be difficult 
or impossible. ByStander incorporates these findings by implementing voice recognition features that allow users to communicate emergency 
details hands-free, addressing scenarios where physical interaction with the device may be limited. \\

Decision-making algorithms in emergency contexts have been explored by \cite{ai_decision_making}. Their research emphasizes 
how AI can effectively navigate complex decision trees to provide optimal recommendations based on multiple variables—a capability 
directly applicable to emergency response scenarios where numerous factors must be considered simultaneously. 
ByStander leverages this approach by implementing decision support algorithms that can rapidly assess emergency type, severity, l
ocation, and available resources to recommend the most appropriate response actions. \\ 

The challenges of complexity in decision-making, particularly under stress, are further examined by \cite{complexity_decision}. 
This analysis highlights how cognitive load increases exponentially with decision complexity, and how this effect is amplified under 
stress—precisely the condition most emergency victims experience. ByStander addresses this challenge by simplifying complex decisions into manageable steps guided by AI, effectively reducing cognitive burden during crisis situations. \\

Regional research specific to Thailand's emergency response systems, such as the study by \cite{thai_ai_emergency}, provides valuable insights into local emergency service infrastructure and challenges. This research informs ByStander's region-specific implementations, ensuring that the application is optimized for Thailand's unique emergency response ecosystem. \\

Finally, practical applications of AI in emergency response settings are documented by \cite{ai_emergency_response}. This analysis showcases successful implementations of AI technologies in emergency dispatch centers, highlighting significant improvements in response times and resource allocation. ByStander builds upon these proven concepts by extending AI assistance directly to users through a mobile interface, creating an end-to-end solution that bridges the gap between emergency victims and professional responders. \\

By synthesizing these research findings, ByStander creates a comprehensive approach to emergency response that addresses both the psychological limitations of users and the practical challenges of emergency service access. The application's integration of AI decision support, voice recognition, location-based routing, and stress-resistant interface design directly applies current research to create a solution that significantly improves emergency outcomes across multiple crisis scenarios. \\
