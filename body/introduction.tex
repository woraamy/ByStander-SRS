\chapter{Introduction}
\label{chap:introduction}

\section{Background}
\label{section:background}

Emergencies strike without warning, and panic often follows. When faced with danger, many people freeze or make poor decisions due to stress, potentially worsening outcomes for themselves and others. This natural stress response can prevent effective action precisely when clear thinking is most crucial.

In Thailand alone, emergency services receive over 4,300 calls daily (1.6 million annually), with traffic accidents (25.6\%), unknown issues (20.4\%), and medical emergencies like abdominal pain (10.6\%) leading the statistics. These situations are particularly dangerous for vulnerable populations like children, the elderly, and those with existing medical conditions, who may require specialized attention during crises.

The consequences of delayed or ineffective emergency responses can be devastating: preventable injuries, loss of life, and lasting psychological trauma. Caregivers and bystanders, despite good intentions, may hesitate or act incorrectly due to emotional distress, leading to preventable harm. This gap between knowledge and action directly impacts survival rates and recovery outcomes.

Our smartphone application addresses this critical need by providing real-time emergency guidance through the technology already in users' pockets. By offering step-by-step instructions during crisis situations, clear direction to reduce panic, and specialized protocols for various emergencies, we can transform emergency response capabilities at the individual level. This research focuses on leveraging widely available mobile technology to improve emergency outcomes when every second counts.

\section{Problem Statement}
\label{section:problem-statement}

In emergency situations, the psychological phenomenon of stress-induced cognitive impairment presents a significant public health challenge. Despite the widespread availability of emergency protocols and safety guidelines, individuals frequently experience severe decision-making paralysis when confronted with high-stress scenarios. This cognitive disconnect—between theoretical knowledge and practical application under pressure—remains inadequately addressed in current emergency response systems.

Observed evidence indicates that approximately 80\% of individuals experience significant mental processing disruptions during high-stress emergencies, manifesting as confusion, memory lapses, and decision-making hesitation. Furthermore, over 60\% of people either freeze completely or make critical errors during emergencies, substantially reducing survival rates and positive outcomes. This phenomenon affects both untrained bystanders and individuals with prior emergency training, suggesting that traditional knowledge-based preparation may be insufficient.

This cognitive impairment is particularly problematic for caregivers of vulnerable populations, including those managing chronic conditions, elderly individuals, and children with special medical needs. The psychological pressure experienced by these caregivers can be even more pronounced, as the consequences of delayed or incorrect actions may be more severe for their dependents.

The gap between emergency knowledge and emergency performance represents a critical area for intervention. While considerable resources have been invested in emergency protocols, comparatively little attention has focused on overcoming the psychological barriers to implementing these protocols during actual emergencies. Our application aims to address this gap by examining the efficacy of real-time digital guidance systems in mitigating stress-induced cognitive impairment and improving emergency response outcomes across diverse scenarios and populations.


\section{Solution Overview}
\label{section:solution-overview}

Bystander is an AI-driven emergency assistance application designed to enhance response efficiency during critical situations. By leveraging real-time location data, the application identifies the most appropriate emergency contact, ensuring faster and more effective assistance. Instead of solely relying on a general emergency hotline, Bystander determines whether contacting local police, a nearby hospital, or specialized emergency services is the best course of action.

The core functionality of the application follows a structured process:
\begin{enumerate}
    \item Incident Detection – When an emergency occurs, the app assists users in identifying the most suitable authority to contact.
    \item Optimized Emergency Call Routing – The application recommends the fastest and most relevant emergency contact based on real-time location data.
    \item Guided Response Actions – While awaiting professional assistance, users receive clear, step-by-step guidance to help manage the situation effectively.
\end{enumerate}

\subsection{Features}
\label{subsection:features}

\begin{enumerate}[leftmargin=80pt]
    \item Real-Time Emergency Service Locator: Utilizes GPS tracking to determine whether calling the general emergency hotline or directly contacting a nearby hospital or police station is the best option.
    \item AI-Generated Emergency Scripts: Automatically compiles a structured emergency report, including key details such as location and the nature of the incident, enabling clear and effective communication with responders.
    \item Step-by-Step Emergency Guidance: Provides easy-to-follow instructions on handling various emergencies, such as administering first aid or assessing an individual’s condition before help arrives.
    \item Community-Powered Guidance: Allows verified experts (e.g., medical professionals, emergency responders) to contribute video or text-based instructional content for different emergency scenarios.
    Content is categorized by emergency type (e.g., fire, traffic accident, medical emergencies) for ease of access.
\end{enumerate}

\section{Target User}
\label{section:target-user}

ByStander is designed for individuals who are at a higher risk of facing emergencies and require immediate assistance in critical situations. The key target users include residents of Thailand who are prone to emergencies, such as those who live with elderly individuals or sick patients who may require urgent medical attention. Or general individuals who have a higher chance of encountering emergencies, including those who frequently drive at night or work in high-risk environments.

- Age Group: 15-60 years old, ensuring accessibility for teenagers, adults, and middle-aged individuals who may need emergency support.

- Skill Level: Users with basic knowledge of technology, ensuring that the application is simple and intuitive for individuals with minimal technical experience.

- Industry or Domain Knowledge: None required, as the application is designed for general use without requiring prior expertise in emergency response or healthcare.


\section{Benefit}
\label{section:benefit}
The app helps people in emergencies by providing faster response times, making it easy to contact the right service quickly. It also ensures clear communication by using AI to create easy-to-understand reports, so users can explain their situation even when they are panicked. The app gives immediate guidance with step-by-step instructions, helping users know what to do while waiting for help. It’s also user-friendly and easy to use, even during stressful situations. The Community Powered Guidance feature offers localized advice from trusted experts, giving users the most relevant and up-to-date information to help them respond effectively in any emergency.


\section{Terminology}
\label{section:terminology}

\begin{enumerate}
    \item Emergency (situation): An unforeseen combination of circumstances or the resulting state that calls for immediate action.
    \item Cognitive Impairment : Problems with a person’s ability to think, learn, remember, use judgement, and make decisions. Signs of cognitive impairment include memory loss and trouble concentrating, completing tasks, understanding, remembering, following instructions, and solving problems. Other common signs may include changes in mood or behavior, loss of motivation, and being unaware of surroundings. Cognitive impairment may be mild or severe. There are many causes of cognitive impairment, including cancer and some cancer treatments.
    \item Geotagging: Adding location information to something, like a picture or a post, so people know where it was taken or where something is happening.
    \item Emergency Hotline: A special phone number you can call for immediate help during an emergency, like calling 911 for urgent situations.
\end{enumerate}